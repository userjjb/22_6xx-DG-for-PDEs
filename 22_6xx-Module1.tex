\documentclass{beamer}
\usepackage[english]{babel}							%For internationalization
\usepackage[utf8]{inputenc}							%For character encoding
\usepackage{amsmath}								%For mathematical typesetting
\usepackage{amssymb}								%For mathematical typesetting
\usepackage{graphicx}								%For handling graphics

\newcommand{\be}{\begin{equation}}
\newcommand{\ben}[1]{\begin{equation}\label{#1}}
\newcommand{\ee}{\end{equation}}

\title
{An Introduction to Discontinuous Galerkin Methods}
\subtitle{Module 1: What is DG?}
\author[Bevan] % (optional, for multiple authors)
{J.~Bevan}
\institute[Universities Here and There] % (optional)
{
  Department of Mechanical Engineering, Grad Student\\
  University of Massachusetts at Lowell
}
\date[Fall 2014] % (optional)
{}
\subject{Discontinuous Galerkin}

\begin{document}
\frame{\titlepage}
\frame{\frametitle{Module 1: What is DG?}\tableofcontents} 

\section{Overall Content Structure}
\subsection{Assumed Prerequisite Knowledge}
\frame{\frametitle{\textbf{\secname}: \subsecname}
\begin{itemize}
\item It is assumed the interested viewer is an advanced undergrad or graduate student with the typical STEM background of Calculus, Linear Algebra, and ODEs/PDEs.
\item Additionally, it is assumed the viewer has at least a basic background in a programming language of their choice (Matlab etc.)
\item Finally it is assumed the viewer has taken a general Numerical Methods course as well a Solution of PDEs course.
\item Not intended to teach common underlying techniques (interpolation etc.), but we may [Recall] important features of them.
\end{itemize} 
}

\subsection{Numerical Methods Prerequisites} 
\frame[shrink]{\frametitle{\subsecname}
\begin{itemize}
\item Linear algebra
	\begin{itemize}
	\item Vector spaces, bases, properties, etc.
	\item Orthogonality
	\item maybe some useful spaces (Hilbert, square integrable, etc)?
	\end{itemize}
\item Polynomial interpolation(1 and 2D)
	\begin{itemize}
	\item Lagrange, Hermite
	\item monomial basis (and ill-conditioned nature)
	\item Orthogonal basis (Legendre, Chebyshev)
	\item L2 projection
	\item choice of interpolation points (equispaced, GL, LGL, etc)
	\item Runge phenomenon
	\item Vandermonde matrix (transformation from modal to nodal spaces)
	\end{itemize} 
\item Quadrature (1 and 2D)
	\begin{itemize}
		\item Newton-Cotes
	\item Gauss/Hermite(Legendre)
	\item relation to interpolation

	\end{itemize} 
\item Solution of ODEs
	\begin{itemize}
	\item Forward Euler
	\item RK4
	\item Implicit schemes (e.g. Backward Euler)
	\item Stability, Convergence
	\end{itemize} 
\end{itemize} 
}

\subsection{Solution of PDEs Prerequisites} 
\frame[shrink]{\frametitle{\subsecname}
\begin{itemize}
\item Domain representation
	\begin{itemize}
	\item meshing
	\item BCs (Neumann and Dirichlet)
	\end{itemize}
\item Finite difference methods (FDM)
	\begin{itemize}
	\item Pointwise spatial derivatives
	\item Computational vs Physical domains
	\item Basic mapping (bilinear)
	\end{itemize} 
\item Finite volume methods (FVM)
	\begin{itemize}
	\item Flux functions
	\item Artificial viscosity
	\item Linear vs nonlinear fluxes
	\end{itemize} 
\item Finite element methods (FEM)
	\begin{itemize}
	\item Weak and strong form formulation
	\item Piecewise linear solution approximation
	\item Galerkin style test functions
	\item local support
	\end{itemize} 
\end{itemize}
}

\subsection{Lecture Goals} 
\frame[shrink]{\frametitle{\subsecname}
\begin{itemize}
\item Understand DG spatial discretization (advective)
\begin{itemize}
\item DG weak form (test function to minimize residual or test function orthogonal)
\item solution approximation (and initial conditions)
\item mapping physical to computation domain (for curvilinear domains)
\item DG Galerkin formulation
\item Integration by parts $\rightarrow$ flux functions (solution smoothness requirements): differences from FEM
\item linear vs non-linear flux: ramifications for semi-discrete system
\item hyperbolic vs parabolic
\item applying BCs (include periodic BCs)
\end{itemize}
\item  Understand time discretization
\begin{itemize}
\item Method of lines style semi-discrete form
\item Types: e.g. Forward Euler, RK4
\item CFL condition and stability
\end{itemize} 
\item Learn how to apply DG to arbitrary PDEs and realm of applicability
\begin{itemize}
\item intuitive understanding of methodology
\item conceptualization of process (not tied down to specific examples)
\item understand pros/cons
\item understand how DG “simplifies” to FVM and FEM
\end{itemize} 
\item Generate runnable code of your own
\begin{itemize}
\item Self-contained set of knowledge and algorithms to be able to write a full solver
\end{itemize} 
\end{itemize}
}

\subsection{Topics Layout} 
\frame{\frametitle{\subsecname}
\textbf{Module 1: What is DG?}\\
DG motivation (why vs FEM, FVM, FDM)\\
Scalar conservation law (linear) PDE\\
Weak form derivation\\
Global domain vs local element\\
Multiple-valued element boundaries\\
Recall: Flux functions\\
}

\frame{\frametitle{\subsecname (cont.)}
\textbf{Module 2: A Simple 1D DG Solver}\\
Linear solution approximation\\
Test function choice (Galerkin)\\
Upwind flux\\
Mass Matrix\\
Stiffness Matrix\\
Putting it all together (linear system)\\
Semi-discrete system\\
Forward Euler\\
Investigate h-convergence\\
Investigate t-convergence\\
Investigate stability (CFL)\\
}

\frame{\frametitle{\subsecname (cont.)}
\textbf{Module 3: To Higher-Orders (nodal) }\\
\textbf{3A: Sol’n Approximation}\\
Revisit weak form\\
-Approx. space\\
-L2 Projection minimizes residual norm\\
-Test space $\rightarrow$ orthogonal\\
Monomial basis?\\
Ill-conditioning of monomials\\
Recall: Lagrange interpolation (code)\\
Derive Lagrange spatial approximation\\
Equispaced interp points?\\
Runge phenomenon\\
Why: Bernstein/Markov inequality\\
Roots of Leg instead\\
}
\frame{\frametitle{\subsecname (cont.)}
\textbf{Module 3: To Higher-Orders (nodal) }\\
\textbf{3B: Discrete System}\\
Numerical Quadrature (Gauss)\\
Hermite interpolation (2N+1 quad)\\
Truncation error/exact quadrature\\
GL Lagrange orthogonality\\
Local Mapping Fun\\
Mass Integral ->diagonal/inversion\\
Log differentiation\\
Flux interpolation\\
Stiffness Integral\\
Numerical Flux (interpolated)\\
Assembly of system\\
RK4 time discretization\\
Investigate p-convergence (smoothness reqs)\\
}

\subsection{A Pedagogical Comment} 
\frame{\frametitle{\subsecname}
\begin{itemize}
\item Take advantage of format: replay, pause, speed up, slow down
\item Each section may have subsections, but the overall section is intended to be a self-contained concept. The first slide of a new section has the title format \textbf{Section}: Subsection
\item Easy to "zone-out", before the start of a new section try and put what you learned into action. Make a code snippet to test your understanding or verify a claimed result etc.
\item Each Module has a larger self-contained concept. You should be able to put together a script that accomplishes something substantial.
\end{itemize} 
}

\section{DG Motivation: Why DG?} 
\frame{\frametitle{\secname}
\begin{itemize}
\item 
\item 
\item 
\end{itemize} 
}

\section{Example PDE: Scalar Conservation Law} 
\frame{\frametitle{\secname}
\begin{itemize}
\item 
\item 
\item 
\end{itemize} 
}

\section{The Weak Form of the PDE} 
\frame{\frametitle{\secname}

}

\section{Domain Decomposition: Global vs Local} 
\frame{\frametitle{\secname}

}

\section{Element Boundaries: Multiply Defined?} 
\frame{\frametitle{\secname}

}

\section{[Recall] Flux Functions} 
\frame{\frametitle{\secname}

}

\end{document}